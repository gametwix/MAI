
\documentclass[pdf, unicode, 12pt, a4paper,oneside,fleqn]{article}
\usepackage{graphicx}
\graphicspath{{img/}}
\usepackage{log-style}
\begin{document}

\begin{titlepage}
    \begin{center}
        \bfseries

        {\Large Московский авиационный институт\\ (национальный исследовательский университет)}
        
        \vspace{48pt}
        
        {\large Факультет информационных технологий и прикладной математики}
        
        \vspace{36pt}
        
        {\large Кафедра вычислительной математики и~программирования}
        
        \vspace{48pt}
        
        Лабораторная работа \textnumero 4 по курсу \enquote{Операционные системы}

        \vspace{48pt}

        Освоение принципов работы с файловыми системами. 
        
        Обеспечение обмена данных между процессами посредством технологии <<File mapping>>.
    \end{center}
    
    \vspace{125pt}
    
    \begin{flushright}
    \begin{tabular}{rl}
    Студент: & П.\,А. Мохляков \\
    Преподаватель: & Е.\,С. Миронов \\
    Группа: & М8О-208Б-19 \\
    Вариант: & 1 \\
    Дата: & \\
    Оценка: & \\
    Подпись: & \\
    \end{tabular}
    \end{flushright}
    
    \vfill
    
    \begin{center}
    \bfseries
    Москва, \the\year
    \end{center}
\end{titlepage}
    
\pagebreak

\section{Постановка задачи}

Составить и отладить программу на языке Си, осуществляющую работу с процессами 
и взаимодействие между ними в одной из двух операционных систем. В результате работы 
программа (основной процесс) должен создать для решения задачи один или несколько 
дочерних процессов. Взаимодействие между процессами осуществляется через системные 
сигналы/события и/или через отображаемые файлы (memory-mapped files).

Необходимо обрабатывать системные ошибки, которые могут возникнуть в результате работы.

Родительский процесс передает команды пользователя дочернему процессу. Дочерний процесс 
принеобходимости передает данные в родительский процесс. Результаты своей работы дочерний
процесс пишет в созданный им файл.

Пользователь вводит команды вида: << число число число<endline> >>. Далее эти числа 
передаются от родительского процесса в дочерний. Дочерний процесс считает их сумму и 
выводит её в файл. Числа имеют тип int.

\section{Сведения о программе}

Программа написанна на Си в Unix подобной операционной системе на базе ядра Linux.
При компиляции следует линковать библиотеки -lpthread и -lrt.
В программе создается дочерний процесс, данные в который передаются с помощью shared memory.

Дочерний прочесс принимает строку чисел и находит их сумму, ответ записывая в файл. Имя файла задается пользователем

Родительский процесс считывает вводные данные у пользоветеля и пердет их дочернему процессу через отброженный участок памати shared memory.

Программа завершает работу при окончании ввода, то есть нажатии CTRL+D.

\section{Общий метод и алгоритм решения}

Сначала в родительском процессе мы создаем shared memory object и получаем их дискрипторы. Первый файл будет 
отвечать за сами данные, второй за их размер, а третий за mutex. Далее мы проецируем
файлы на память, и инициазируем. Создав перед этим атрибуты для мютекса, в частности для того чтобы он работал для разных процессов.

Далее мы делаем fork() и запускаем в ребенке его программу, передав как оргументы имя результирующего файла,
и имена рбъектов shared memory. В дочернем процессе мы также открываем shared memory и проецируем их.

Потом идет логика самой программы. В начале родитель блокирует mutex, считывает счисло и символ за ним, добавляя число в спроецировннй массив и увличивая счетчик его длины.
Если символ за числом будет равен символу конца строки, то мы разблокируем mutex и ожидаем пока 
переменная длины массива не станет равна нулю, далее блокируем muteх и повторяем цикл. При окончании ввода родитель присваивает 
переменной длине массива значение $-1$ и разблокируем mutex.

Ребенок в это время также пытается заблокировать mutex, если у него это получается до ввода родителя, то 
он его сразу разблокирует и повторяет цикл. Если же ввод был совершен и переменная длины массива больше нуля,
то мы считываем Size елементов в массиве и суммируем их, записывая ответ в файл, разблокируем mutex и повторяем цикл.
В случае если длина массива данных станет $-1$, то мы выходим из цикла.

Далее мы в обоих программах отключаем проекции и закрываем файлы.


\section{Листинг программы}

{\large\textbf{parent.c}}

\begin{lstlisting}[language=C]
#include <stdio.h>
#include <string.h>
#include <unistd.h>
#include <sys/mman.h>
#include <sys/types.h>
#include <fcntl.h>
#include <pthread.h>

#define SH_NAME "my_shared_mem"
#define SH_SIZE_NAME "my_shared_mem_size"
#define MUTEX_NAME "my_mutex"

void wait(int *elem, int num){
    while (*elem != num){
    }
}

int main()
{
    int fd_shared_data = shm_open(SH_NAME, O_RDWR | O_CREAT, S_IRWXU);
    int fd_shared_data_size = shm_open(SH_SIZE_NAME, O_RDWR | O_CREAT, S_IRWXU);
    int fd_mutex = shm_open(MUTEX_NAME,O_RDWR | O_CREAT, S_IRWXU);

    if(fd_shared_data == -1 || fd_shared_data_size == -1 || fd_mutex == -1){
        printf("Error: shared memory open\n");
        return -1;
    }
    if(ftruncate(fd_shared_data,getpagesize()) == -1){
        printf("Error: ftruncate\n");
        return -1;
    }
    if(ftruncate(fd_shared_data_size,sizeof(int)) == -1){
        printf("Error: ftruncate\n");
        return -1;
    }
    if(ftruncate(fd_mutex,sizeof(pthread_mutex_t*)) == -1){
        printf("Error: ftruncate\n");
        return -1;
    }

    int *Data = (int*) mmap(NULL,getpagesize(),PROT_READ | PROT_WRITE, MAP_SHARED, fd_shared_data, 0);
    int *Size = (int*) mmap(NULL,sizeof(int),PROT_READ | PROT_WRITE, MAP_SHARED, fd_shared_data_size, 0);
    pthread_mutex_t *Lock = (pthread_mutex_t*) mmap(NULL,sizeof(pthread_mutex_t*),PROT_READ | PROT_WRITE, MAP_SHARED,fd_mutex,0);
    if (Data == MAP_FAILED || Size == MAP_FAILED || Lock  == MAP_FAILED) {
        printf("Error: map file\n");
        return -1;
    }
    pthread_mutexattr_t MutexAttribute;
    if(pthread_mutexattr_setpshared(&MutexAttribute, PTHREAD_PROCESS_SHARED) != 0){
        printf("Error: set shared attribute mutex\n");
        return -1;
    }
    *Size = 0;
    if(pthread_mutex_init(Lock, &MutexAttribute) != 0){
        printf("Error: mutex init\n");
        return -1;
    }    

    char *filename = NULL;
    size_t sizename = 0;
    getline(&filename,&sizename,stdin);
    filename[strlen(filename)-1] = '\0';

    int id = fork();

    if(id == -1){
        printf("Error: fork\n");
        return -1;
    } else if(id == 0) {

        execl("./child","child",filename,SH_NAME,SH_SIZE_NAME,MUTEX_NAME,(char*) NULL);
    } else {

        int num;
        char sym;
        if(pthread_mutex_lock(Lock) != 0){
            printf("Error: mutex lock\n");
            return -1;
        }
        while(scanf("%d%c",&num,&sym) > 0){
            Data[*Size] = num;
            *Size += 1;
            if(sym == '\n'){
                if(pthread_mutex_unlock(Lock) != 0){
                    printf("Error: mutex unlock\n");
                    return -1;
                }
                wait(Size,0);
                if(pthread_mutex_lock(Lock) != 0){
                    printf("Error: mutex lock\n");
                    return -1;
                }
            }
        }
        *Size = -1;
        if(pthread_mutex_unlock(Lock) != 0){
            printf("Error: mutex unlock\n");
            return -1;
        }
    }

    if(munmap(Data,getpagesize()) != 0){
        printf("Error: unmap file\n");
        return -1;
    }
    if(munmap(Size,sizeof(int)) != 0){
        printf("Error: unmap file\n");
        return -1;
    }
    if(munmap(Lock,sizeof(pthread_mutex_t*)) != 0){
        printf("Error: unmap file\n");
        return -1;
    }
    if(close(fd_shared_data) < 0){
        printf("Error: close file\n");
        return -1;
    }
    if(close(fd_shared_data_size) < 0){
        printf("Error: close file\n");
        return -1;
    }
    if(close(fd_mutex) < 0){
        printf("Error: close file\n");
        return -1;
    }

    return 0;
}
\end{lstlisting}

{\large\textbf{child.c}}

\begin{lstlisting}[language=C]
#include "stdio.h"
#include <unistd.h>
#include <sys/mman.h>
#include <sys/types.h>
#include <fcntl.h>
#include <pthread.h>


int main(int argc,char **argv){
    if(argc < 5){
        printf("Arguments error");
        return 1;
    }
    char *filename = argv[1];
    char *sh_data_name = argv[2];
    char *sh_data_size_name = argv[3];
    char *mutex_name = argv[4];

    int fd_shared_data = shm_open(sh_data_name, O_RDWR | O_CREAT, S_IRWXU);
    int fd_shared_data_size = shm_open(sh_data_size_name, O_RDWR | O_CREAT, S_IRWXU);
    int fd_mutex = shm_open(mutex_name,O_RDWR | O_CREAT, S_IRWXU);
    if(fd_shared_data == -1 || fd_shared_data_size == -1 || fd_mutex == -1){
        printf("Error: shared memory open\n");
        return -1;
    }
    int *Data = (int*) mmap(NULL,getpagesize(),PROT_READ | PROT_WRITE, MAP_SHARED, fd_shared_data, 0);
    int *Size = (int*) mmap(NULL,sizeof(int),PROT_READ | PROT_WRITE, MAP_SHARED, fd_shared_data_size, 0);
    pthread_mutex_t *Lock = (pthread_mutex_t*) mmap(NULL,sizeof(pthread_mutex_t*),PROT_READ | PROT_WRITE, MAP_SHARED,fd_mutex,0);
    if (Data == MAP_FAILED || Size == MAP_FAILED || Lock  == MAP_FAILED) {
        printf("Error: map file\n");
        return -1;
    }
    FILE *file;
    file = fopen(filename,"w");
    if(file == NULL){
        printf("Error: fopen file\n");
        return -1;
    }
    while ((*Size) != -1){
        if(pthread_mutex_lock(Lock) != 0){
            printf("Error: mutex lock\n");
            return -1;
        }
        if(*Size > 0){
            long long sum = 0;
            for(int i = 0;i < *Size; ++i){
                sum += Data[i];
            }
            *Size = 0;
            fprintf(file,"%lld\n",sum);
        }
        if(pthread_mutex_unlock(Lock) != 0){
            printf("Error: mutex unlock\n");
            return -1;
        }
    }

    if(fclose(file) != 0){
        printf("Error: fclose file\n");
        return -1;
    }
    if(munmap(Data,getpagesize()) != 0){
        printf("Error: unmap file\n");
        return -1;
    }
    if(munmap(Size,sizeof(int)) != 0){
        printf("Error: unmap file\n");
        return -1;
    }
    if(munmap(Lock,sizeof(pthread_mutex_t*)) != 0){
        printf("Error: unmap file\n");
        return -1;
    }
    if(close(fd_shared_data) < 0){
        printf("Error: close file\n");
        return -1;
    }
    if(close(fd_shared_data_size) < 0){
        printf("Error: close file\n");
        return -1;
    }
    if(close(fd_mutex) < 0){
        printf("Error: close file\n");
        return -1;
    }
    if(shm_unlink(sh_data_name) != 0){
        printf("Error: shared memory unlink\n");
        return -1;
    }
    if(shm_unlink(sh_data_size_name) != 0){
        printf("Error: shared memory unlink\n");
        return -1;
    }
    if(shm_unlink(mutex_name)!= 0){
        printf("Error: shared memory unlink\n");
        return -1;
    }
    return 0;
}
\end{lstlisting}

\section{Демонстрация работы программы}

\begin{alltt}
pavel@DESKTOP-K5KMLPV:~/Project/mai/2_course/OS/LB4$ make
gcc -c -Wall parent.c
gcc parent.o -pthread -lrt -o parent
gcc -c -Wall child.c
gcc child.o -pthread -lrt -o child
pavel@DESKTOP-K5KMLPV:~/Project/mai/2_course/OS/LB4$ ./parent
test
1 2 3 4 5
0 0 0
12 45 34 54
42 -5
pavel@DESKTOP-K5KMLPV:~/Project/mai/2_course/OS/LB4$ cat test
15
0
145
37
pavel@DESKTOP-K5KMLPV:~/Project/mai/2_course/OS/LB4$ strace -f -e trace=
"%process,read,write,dup2,mmap" -o log.txt ./parent
test2
1 2 3
0 0
2 -1
pavel@DESKTOP-K5KMLPV:~/Project/mai/2_course/OS/LB4$ cat log.txt
679   execve("./parent", ["./parent"], 0x7fffe2329438 /* 29 vars */) = 0
679   arch_prctl(0x3001 /* ARCH_??? */, 0x7ffe9622c3c0) = -1 
EINVAL (Invalid argument)
679   mmap(NULL, 45372, PROT_READ, MAP_PRIVATE, 3, 0) = 0x7f8283ee3000
679   read(3, "\textbackslash177ELF\textbackslash2\textbackslash1\textbackslash1\textbackslash0\textbackslash0\textbackslash0\textbackslash0\textbackslash0\textbackslash0\textbackslash0\textbackslash0\textbackslash0\textbackslash3\textbackslash0>\textbackslash0\textbackslash1\textbackslash0\textbackslash0\textbackslash0 7\textbackslash0
\textbackslash0\textbackslash0\textbackslash0\textbackslash0\textbackslash0"..., 832) = 832
679   mmap(NULL, 8192, PROT_READ|PROT_WRITE, MAP_PRIVATE
|MAP_ANONYMOUS, -1, 0) = 0x7f8283ee1000
679   mmap(NULL, 44000, PROT_READ, MAP_PRIVATE|
MAP_DENYWRITE, 3, 0) = 0x7f8283ed6000
679   mmap(0x7f8283ed9000, 16384, PROT_READ|PROT_EXEC, MAP_PRIVATE|MAP_FIXED
|MAP_DENYWRITE, 3, 0x3000) = 0x7f8283ed9000
679   mmap(0x7f8283edd000, 4096, PROT_READ, MAP_PRIVATE|MAP_FIXED|
MAP_DENYWRITE, 3, 0x7000) = 0x7f8283edd000
679   mmap(0x7f8283edf000, 8192, PROT_READ|PROT_WRITE, MAP_PRIVATE|
MAP_FIXED|MAP_DENYWRITE, 3, 0x8000) = 0x7f8283edf000
679   read(3, "\textbackslash177ELF\textbackslash2\textbackslash1\textbackslash1\textbackslash0\textbackslash0\textbackslash0\textbackslash0\textbackslash0\textbackslash0\textbackslash0\textbackslash0\textbackslash0\textbackslash3\textbackslash0>\textbackslash0\textbackslash1\textbackslash0\textbackslash0\textbackslash0\textbackslash220
\textbackslash201\textbackslash0\textbackslash0\textbackslash0\textbackslash0\textbackslash0\textbackslash0"..., 832) = 832
679   mmap(NULL, 140408, PROT_READ, MAP_PRIVATE|
MAP_DENYWRITE, 3, 0) = 0x7f8283eb3000
679   mmap(0x7f8283eba000, 69632, PROT_READ|PROT_EXEC, MAP_PRIVATE|
MAP_FIXED|MAP_DENYWRITE, 3, 0x7000) = 0x7f8283eba000
679   mmap(0x7f8283ecb000, 20480, PROT_READ, MAP_PRIVATE|MAP_FIXED|
MAP_DENYWRITE, 3, 0x18000) = 0x7f8283ecb000
679   mmap(0x7f8283ed0000, 8192, PROT_READ|PROT_WRITE, MAP_PRIVATE|
MAP_FIXED|MAP_DENYWRITE, 3, 0x1c000) = 0x7f8283ed0000
679   mmap(0x7f8283ed2000, 13432, PROT_READ|PROT_WRITE, MAP_PRIVATE|
MAP_FIXED|MAP_ANONYMOUS, -1, 0) = 0x7f8283ed2000
679   read(3, "\textbackslash177ELF\textbackslash2\textbackslash1\textbackslash1\textbackslash3\textbackslash0\textbackslash0\textbackslash0\textbackslash0\textbackslash0\textbackslash0\textbackslash0\textbackslash0\textbackslash3\textbackslash0>\textbackslash0\textbackslash1\textbackslash0\textbackslash0\textbackslash0\textbackslash360q
\textbackslash2\textbackslash0\textbackslash0\textbackslash0\textbackslash0\textbackslash0"..., 832) = 832
679   mmap(NULL, 2036952, PROT_READ, MAP_PRIVATE|
MAP_DENYWRITE, 3, 0) = 0x7f8283cc1000
679   mmap(0x7f8283ce6000, 1540096, PROT_READ|PROT_EXEC, MAP_PRIVATE|
MAP_FIXED|MAP_DENYWRITE, 3, 0x25000) = 0x7f8283ce6000
679   mmap(0x7f8283e5e000, 303104, PROT_READ, MAP_PRIVATE|MAP_FIXED|
MAP_DENYWRITE, 3, 0x19d000) = 0x7f8283e5e000
679   mmap(0x7f8283ea9000, 24576, PROT_READ|PROT_WRITE, MAP_PRIVATE|
MAP_FIXED|MAP_DENYWRITE, 3, 0x1e7000) = 0x7f8283ea9000
679   mmap(0x7f8283eaf000, 13528, PROT_READ|PROT_WRITE, MAP_PRIVATE|
MAP_FIXED|MAP_ANONYMOUS, -1, 0) = 0x7f8283eaf000
679   mmap(NULL, 12288, PROT_READ|PROT_WRITE, MAP_PRIVATE|
MAP_ANONYMOUS, -1, 0) = 0x7f8283cbe000
679   arch_prctl(ARCH_SET_FS, 0x7f8283cbe740) = 0
679   mmap(NULL, 4096, PROT_READ|PROT_WRITE, 
MAP_SHARED, 3, 0) = 0x7f8283f1b000
679   mmap(NULL, 4, PROT_READ|PROT_WRITE, MAP_SHARED, 4, 0) = 0x7f8283eee000
679   mmap(NULL, 8, PROT_READ|PROT_WRITE, MAP_SHARED, 5, 0) = 0x7f8283eed000
679   read(0, "test2\textbackslash{n}", 1024)          = 6
679   clone(child_stack=NULL, flags=CLONE_CHILD_CLEARTID|
CLONE_CHILD_SETTID|SIGCHLD, child_tidptr=0x7f8283cbea10) = 680
679   read(0,  <unfinished ...>
680   execve("./child", ["child", "test2", "my_shared_mem",
"my_shared_mem_size","my_mutex"], 0x7ffe9622c4a8 /* 29 vars */) = 0
680   arch_prctl(0x3001 /* ARCH_??? */, 0x7ffc20777f70) = -1 
EINVAL (Invalid argument)
680   mmap(NULL, 45372, PROT_READ, MAP_PRIVATE, 3, 0) = 0x7fa203107000
680   read(3, "\textbackslash177ELF\textbackslash2\textbackslash1\textbackslash1\textbackslash0\textbackslash0\textbackslash0\textbackslash0\textbackslash0\textbackslash0\textbackslash0\textbackslash0\textbackslash0\textbackslash3\textbackslash0>\textbackslash0\textbackslash1\textbackslash0\textbackslash0\textbackslash0 7\textbackslash0
\textbackslash0\textbackslash0\textbackslash0\textbackslash0\textbackslash0"..., 832) = 832
680   mmap(NULL, 8192, PROT_READ|PROT_WRITE, MAP_PRIVATE|
MAP_ANONYMOUS, -1, 0) = 0x7fa203105000
680   mmap(NULL, 44000, PROT_READ, MAP_PRIVATE|
MAP_DENYWRITE, 3, 0) = 0x7fa2030fa000
680   mmap(0x7fa2030fd000, 16384, PROT_READ|PROT_EXEC, MAP_PRIVATE|
MAP_FIXED|MAP_DENYWRITE, 3, 0x3000) = 0x7fa2030fd000
680   mmap(0x7fa203101000, 4096, PROT_READ, MAP_PRIVATE|MAP_FIXED|
MAP_DENYWRITE, 3, 0x7000) = 0x7fa203101000
680   mmap(0x7fa203103000, 8192, PROT_READ|PROT_WRITE, MAP_PRIVATE|
MAP_FIXED|MAP_DENYWRITE, 3, 0x8000) = 0x7fa203103000
680   read(3, "\textbackslash177ELF\textbackslash2\textbackslash1\textbackslash1\textbackslash0\textbackslash0\textbackslash0\textbackslash0\textbackslash0\textbackslash0\textbackslash0\textbackslash0\textbackslash0\textbackslash3\textbackslash0>\textbackslash0\textbackslash1\textbackslash0\textbackslash0\textbackslash0\textbackslash220
\textbackslash201\textbackslash0\textbackslash0\textbackslash0\textbackslash0\textbackslash0\textbackslash0"..., 832) = 832
680   mmap(NULL, 140408, PROT_READ, MAP_PRIVATE|
MAP_DENYWRITE, 3, 0) = 0x7fa2030d7000
680   mmap(0x7fa2030de000, 69632, PROT_READ|PROT_EXEC, MAP_PRIVATE|
MAP_FIXED|MAP_DENYWRITE, 3, 0x7000) = 0x7fa2030de000
680   mmap(0x7fa2030ef000, 20480, PROT_READ, MAP_PRIVATE|MAP_FIXED|
MAP_DENYWRITE, 3, 0x18000) = 0x7fa2030ef000
680   mmap(0x7fa2030f4000, 8192, PROT_READ|PROT_WRITE, MAP_PRIVATE|
MAP_FIXED|MAP_DENYWRITE, 3, 0x1c000) = 0x7fa2030f4000
680   mmap(0x7fa2030f6000, 13432, PROT_READ|PROT_WRITE, MAP_PRIVATE|
MAP_FIXED|MAP_ANONYMOUS, -1, 0) = 0x7fa2030f6000
680   read(3, "\textbackslash177ELF\textbackslash2\textbackslash1\textbackslash1\textbackslash3\textbackslash0\textbackslash0\textbackslash0\textbackslash0\textbackslash0\textbackslash0\textbackslash0\textbackslash0\textbackslash3\textbackslash0>\textbackslash0\textbackslash1\textbackslash0\textbackslash0\textbackslash0\textbackslash360q
\textbackslash2\textbackslash0\textbackslash0\textbackslash0\textbackslash0\textbackslash0"..., 832) = 832
680   mmap(NULL, 2036952, PROT_READ, MAP_PRIVATE|MAP_DENYWRITE, 3, 0)
 = 0x7fa202ee5000
680   mmap(0x7fa202f0a000, 1540096, PROT_READ|PROT_EXEC, MAP_PRIVATE|
MAP_FIXED|MAP_DENYWRITE, 3, 0x25000) = 0x7fa202f0a000
680   mmap(0x7fa203082000, 303104, PROT_READ, MAP_PRIVATE|MAP_FIXED|
MAP_DENYWRITE, 3, 0x19d000) = 0x7fa203082000
680   mmap(0x7fa2030cd000, 24576, PROT_READ|PROT_WRITE, MAP_PRIVATE|
MAP_FIXED|MAP_DENYWRITE, 3, 0x1e7000) = 0x7fa2030cd000
680   mmap(0x7fa2030d3000, 13528, PROT_READ|PROT_WRITE, MAP_PRIVATE|
MAP_FIXED|MAP_ANONYMOUS, -1, 0) = 0x7fa2030d3000
680   mmap(NULL, 12288, PROT_READ|PROT_WRITE, MAP_PRIVATE|
MAP_ANONYMOUS, -1, 0) = 0x7fa202ee2000
680   arch_prctl(ARCH_SET_FS, 0x7fa202ee2740) = 0
680   mmap(NULL, 4096, PROT_READ|PROT_WRITE, 
MAP_SHARED, 3, 0) = 0x7fa20313f000
680   mmap(NULL, 4, PROT_READ|PROT_WRITE, MAP_SHARED, 4, 0) = 0x7fa203112000
680   mmap(NULL, 8, PROT_READ|PROT_WRITE, MAP_SHARED, 5, 0) = 0x7fa203111000
679   <... read resumed>"1 2 3\textbackslash{n}", 1024) = 6
679   read(0, "0 0\textbackslash{n}", 1024)            = 4
679   read(0, "2 -1\textbackslash{n}", 1024)           = 5
679   read(0, "", 1024)                 = 0
680   write(6, "6\textbackslash{n}0\textbackslash{n}1\textbackslash{n}", 6)          = 6
679   exit_group(0)                     = ?
679   +++ exited with 0 +++
680   exit_group(0)                     = ?
680   +++ exited with 0 +++
\end{alltt}

\pagebreak

\section{Вывод}

Взаимодействие между процессами можно организовать при помощи каналов, сокетов и отображаемых
файлов. В данной лабораторной работе был изучен и применен механизм межпроцессорного взаимодействия 
\--- file mapping. Файл отображается на оперативную память таким образом, что мы можем взаимодействовать
с ним как с массивом.

Благодаря этому вместо медленных запросов на чтение и запись мы выполняем отображение файла в
ОЗУ и получаем произвольный доступ за O(1). Из-за этого при использовании этой технологии межпроцессорного
взаимодействия мы можем получить ускорении работы программы, в сравнении, с использованием каналов.

Из недостатков данного метода можно выделить то, что дочерние процессы обязательно должны знать
имя отображаемого файла и также самостоятельно выполнить отображение.
\end{document}



\CWHeader{Лабораторная работа \textnumero 3}

\CWProblem{
Для реализации словаря из предыдущей лабораторной работы, необходимо провестиисследование скорости выполнения и потребления оперативной памяти. В случаевыявления ошибок или явных недочётов, требуется их исправить.

Результатом лабораторной работы является отчёт, состоящий из:

\begin{itemize}
    \item Дневника выплонения работы, в котором отражено что и когда делалось, какиесредства использовались и какие результаты были достигнуты на каждом шаге вы-полнения лабораторной работы.
    \item Выводов о найденных недочётах.
    \item Сравнение работы исправленной программы с предыдущей версией.
    \item Общих выводов о выполнении лабораторной работы, полученном опыте.
\end{itemize}

Минимальный набор используемых средств должен содержать утилиту gprof и биб-лиотеку dmalloc, однако их можно заменять на любые другие аналогичные или болееразвитые утилиты (например, Valgrind или Shark) или добавлять к ним новые (на-пример, gcov)

{\bfseries Используемые утилиты:} Valgrind,Callgrind,QCachegrind.
}
\pagebreak

\section{Описание}
Строки фиксированной длины 64 символа, во входных данных могут встретиться строки меньшей длины, при этом строка дополняется до 64-х нулевыми символами, которые не выводятся на экран.

\pagebreak

\section{Исходный код}
Для начала напишем шаблонный класс вектора, полями вектора являются динамический массив, количество занятых ячеек и общее количество ячеек. При добавлении нового элимента в конец массива мы проверярем, есть ли в массиве еще место, если его нет, то увеличивает массив в 2 раза.


Для описания элементов массива я создал класс, хранящий в себе пару автомабольный номер и строка. И описал для него операцию присваевания и сравнения. Операция сравнения необходима для сравнения скорости моей сортировки и стандартной сортировки слиянием.


Далее мы считываем все данные которые нам доступны из потока, и помещаем их в вектор.


Сортировка разбита на 2 функции. Первая сортирует элементы вектора по возврастанию буквы в алфавитном порядке, вторая по возврастанию чисел от нуля до тысячи. Для ускорения работы программы и сокращения количество копирований мы создаем копию вектора той же длины, что и исходный, в который будет складываться ответ. После чего для следующей функции уже второй вектор будет исходным, а в первый будет складываться ответ. Таким образом мы их меняем местами нужное количество раз.


Сами сортировки являются сортировками подсчета. То есть мы создаем массим состоящий из количества элиментов в разряде. После чего считаем сколько раз на встетился тот, или иной элимент. Потом мы прибавляем к каждому элименту этого массива сумму всех предидущих элиментов, тем самым получая положения последнего элемента вектора для каждого значения разряда. Далее мы с конца проходим начальный вектор и копируем значения его элимента в ячейку второго вектора, номер которой задан в массиве. После добавления элимента мы уменьшаем на 1 значение в этой ячейке массива.

\pagebreak

\begin{lstlisting}[language=C]
#ifndef VECTOR
#define VECTOR

#include <cstring>
#include <cstddef>
#include <iostream>

namespace NVector
{
    int min(int a,int b);
    
    
    template <typename T>
    class TVector
    {
        protected:
        T *Data;
        size_t Size_data;
        size_t Size_malloc;

        void Resize(size_t newsize);
        
        void Copy(T*& data1, T*& data2, size_t begin, size_t end);
    
        void Clear();
        
        public:

        TVector();

        ~TVector();

        const int Size();

        int Size();

        void Renew(size_t new_size);

        T* Begin();

        T* End();

        void Push_back(const T& item);

        T& operator[](size_t itr);

        const T &operator[](size_t itr);

        void operator=(const TVector<T> &second); 

    }; //class TVector

} //namespace NVector

#endif

#ifndef SORTBASE
#define SORTBASE

#include "vector.hpp"
#include "base.hpp"

namespace NSort_base
{
    const void Sort_ch( NVector::TVector<NBase::TBase_elem>  &base,NVector::TVector<NBase::TBase_elem> &newbase, int num)
    {
        int k[26] = {};
        //newbase.Renew(base.Size());
        int size = base.Size();
        for(int i = 0;i < size;++i)
        {
            int l = (int)base[i].Carnum.Sym[num]-65;
            ++k[l];
        }
        for(int i=1;i<26;++i)
        {
            k[i]+=k[i-1];
        }
        for(int i = size - 1;i >= 0;--i)
        {
            int l = (int)base[i].Carnum.Sym[num]-65;
            newbase[k[l]-1] = base[i];
            --k[l];
        }

    }


    void Sort_int(const NVector::TVector<NBase::TBase_elem> &base,NVector::TVector<NBase::TBase_elem> &newbase)
    {
        int k[1000] = {};
        //newbase.Renew(base.Size());
        int size = base.Size();
        for(int i = 0;i < size;++i)
        {
            int l = base[i].Carnum.Num;
            ++k[l];
        }
        for(int i=1;i<1000;++i)
        {
            k[i]+=k[i-1];
        }
        for(int i = size - 1;i >= 0;--i)
        {
            int l = base[i].Carnum.Num;
            newbase[k[l]-1] = base[i];
            --k[l];
        }
    }

    void Sort(NVector::TVector<NBase::TBase_elem> &base1)
    {
        NVector::TVector<NBase::TBase_elem> base2;
        base2.Renew(base1.Size());
        Sort_ch(base1,base2,2);
        Sort_ch(base2,base1,1);
        Sort_int(base1,base2);
        Sort_ch(base2,base1,0);
    }

} // namespace NSort_base

#endif

#ifndef BASE
#define BASE

#include <iostream>
#include <iomanip>
#include <cstring>

namespace NCarnum
{
    class TCarnum
    {
        public:
        int Num = 0;
        char Sym[3]={'A','A','A'};

        void operator=(const TCarnum &second_elem);

        friend std::istream& operator>>(std::istream &in, TCarnum &carnum);

        friend std::ostream& operator<< (std::ostream &out, const TCarnum &carnum);

        bool operator<(const TCarnum &b);
        
    }; // class TCarnum

} // namespace NCarnum


namespace NBase
{
    class TBase_elem
    {
        public:
        NCarnum::TCarnum Carnum;
        char Str[65] = "";

        void operator=(const TBase_elem &second_elem);

        friend std::istream& operator>>(std::istream &in, TBase_elem &elem);

        friend std::ostream& operator<< (std::ostream &out, const TBase_elem &elem);

        bool operator< (const TBase_elem &b);
    }; // class TCarnum
} // namespace NCarnum

#endif
\end{lstlisting}

\pagebreak

\begin{longtable}{|p{7.5cm}|p{7.5cm}|}
\hline
\rowcolor{lightgray}
\multicolumn{2}{|c|} {main.cpp}\\
\hline
int main()&Считывает данные в вектор, сортирует и выводит\\
\hline
\rowcolor{lightgray}
\multicolumn{2}{|c|} {vector.hpp}\\
\hline
void Resize(size\_t newsize)&Меняет размер массива с данными\\
\hline
void Copy(T*\& data1, T*\& data2, size\_t begin, size\_t end)&Копирует данные из одного массива в другой\\
\hline
void Clear()&Очищает массив\\
\hline
int Size()&Возвращает размер массива\\
\hline
void Renew(size\_t new\_size)&Изменяет размер массива и записывает этот размер в поля\\
\hline
T* Begin()&Возвращает указатель на первый элимент массива\\
\hline
T* End()&Возвращает указатель на элимент массива после последнего\\
\hline
void Push\_back(const T\& item)&Добавляет элимент в конец вектора, если это необходимо, то меняет его размер\\
\hline
\rowcolor{lightgray}
\multicolumn{2}{|c|} {sort\_base.hpp}\\
\hline
const void Sort\_ch( NVector:: TVectorNBase:: TBase\_elem \&base,NVector:: TVectorNBase:: TBase\_elem \&newbase, int num)&Сортирует элименты по определённой букве в номере\\
\hline
void Sort\_int(const NVector:: TVectorNBase:: TBase\_elem \&base,NVector:: TVectorNBase:: TBase\_elem \&newbase)&Сортирует элименты по числу\\
\hline
void Sort(NVector:: TVectorNBase:: TBase\_elem \&base1)&	Вызывает сортировки в нужном порядке\\
\hline
\end{longtable}

\pagebreak

\section{Консоль}
\begin{alltt}
 pavel@pc ~/Projects/mai/2_course/DA/LB1/solution cat tests/test_1.t 
A 159 PW	vnooqchksmimoghvylzenlpinabmekwqtgkrlrobqslxysucapzhyoeplbjtszge
G 265 SA	xjcwlteiidgberucfaqegemae
L 431 HA	vysiuauvmpgdyalgccevgaikpwjzkhhftblzhdkcqtszivdmhm
L 759 JJ	kdcpzfqzhqlozuvnutrdlhsnfkiqpqbisafzeivoyfrkvgwunbqcyzpimsi
O 072 VE	dduvzhnfhvdeymgfgu
O 511 TC	gqsgphojrlkdz
Q 902 LQ	omypycnyvcibio
T 916 ZH	inezpvokworktbaobbdlbejhqjvipyqdbslecedrasduhllavkiknc
U 482 CI	iregithcmyzujjqkjdpqqmfdzjaelsqdnhjqobutpojrl
V 084 BM	fhoobacktutcvmbvbrcileqybzlczytlleucqnlnpbyiyblsf
W 031 GY	wovxkbohipmgdlrabpvaeoznuxkvhitehlvsbgxafghe
 pavel@pc ~/Projects/mai/2_course/DA/LB1/solution  g++ main.cpp -o main
 pavel@pc ~/Projects/mai/2_course/DA/LB1/solution  cat tests/test_1.t | ./main 
A 159 PW	vnooqchksmimoghvylzenlpinabmekwqtgkrlrobqslxysucapzhyoeplbjtszge
G 265 SA	xjcwlteiidgberucfaqegemae
L 431 HA	vysiuauvmpgdyalgccevgaikpwjzkhhftblzhdkcqtszivdmhm
L 759 JJ	kdcpzfqzhqlozuvnutrdlhsnfkiqpqbisafzeivoyfrkvgwunbqcyzpimsi
O 072 VE	dduvzhnfhvdeymgfgu
O 511 TC	gqsgphojrlkdz
Q 902 LQ	omypycnyvcibio
T 916 ZH	inezpvokworktbaobbdlbejhqjvipyqdbslecedrasduhllavkiknc
U 482 CI	iregithcmyzujjqkjdpqqmfdzjaelsqdnhjqobutpojrl
V 084 BM	fhoobacktutcvmbvbrcileqybzlczytlleucqnlnpbyiyblsf
W 031 GY	wovxkbohipmgdlrabpvaeoznuxkvhitehlvsbgxafghe
\end{alltt}
\pagebreak

